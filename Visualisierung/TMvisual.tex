\documentclass[10pt]{beamer} 
 \hypersetup{pdfpagemode=FullScreen} 
 \usepackage{tikz} 
 \usetikzlibrary{shadows,patterns,shapes} 
 \usetikzlibrary{shapes.arrows,chains} 
 % serifenfreier Font -- fuer Praesentation geeignet/er 
\renewcommand\familydefault{\sfdefault} 
 \listfiles % damit im Log alle benutzten Pakete aufgelistet werden 
 \usetheme[progressbar=frametitle]{metropolis} 
 \usepackage{appendixnumberbeamer} 
 \usepackage{booktabs} 
 \usepackage[scale=2]{ccicons} 
 \usepackage[utf8]{inputenc} 
 \usepackage{pgfplots} 
 \usepgfplotslibrary{dateplot} 
 \usepackage[ngerman]{babel} 
 \usepackage{xspace} 
 \newcommand{\themename}{\textbf{\textsc{metropolis}}\xspace} 
 \title{Deterministische Turing-Maschine} 
 \subtitle{TU-Berlin, SoSe17, CoMa2 Programmierprojekt} 
 \author{Carolin Schwarz 371802 CO2-155,\\ Duc Hoang Tran 222476 CO2-133,\\ Li Yinying 380390 CO2-144\\} 
 \institute{Betreuer: Ansgar} 
 \begin{document} 
 \maketitle 
\begin{frame}[fragile]{CoMa Turingmaschine} 
 
 \begin{itemize} 
 \item Eingabealphabet : ['11'] 
 \item Bandalphabet : [1, 2, 3] 
 \item Leerzeichen : B 
 \item anzahl an Zustaenden : 2 
 \item akzept. Zustaende : q1,q2
 \end{itemize} 
 \begin{figure} 
 \begin{tikzpicture} 
 
 \edef\sizetape{0.7cm} 
 \tikzstyle{tmtape}=[draw,minimum size=\sizetape] 
 
 %% Draw TM tape 
 \begin{scope}[start chain=1 going right,node distance=-0.15mm] 
 \node [on chain=1,tmtape,draw=none] {$\ldots$}; 
 \node [on chain=1,tmtape] {B}; 
 \node [on chain=1,tmtape] {B}; 
 \node [on chain=1,tmtape] {B}; 
 \node [on chain=1,tmtape] {21}; 
 \node [on chain=1,tmtape] {78}; 
 \node [on chain=1,tmtape] {B}; 
 \node [on chain=1,tmtape] {B}; 
 \node [on chain=1,tmtape] {B}; 
 \node [on chain=1,tmtape] {B}; 
 \node [on chain=1,tmtape] {B}; 
 \node [on chain=1,tmtape] {B}; 
 \node [on chain=1,tmtape] {B}; 
 \node [on chain=1,tmtape] {B}; 
  
 \node [on chain=1,tmtape,draw=none] {$\ldots$}; 
 \end{scope} 
 \end{tikzpicture} 
 \begin{tikzpicture} 
 \node [draw,align=left]{akt. Zustand}; 
 \begin{scope}[start chain=2 going right] 
 \node [draw,left=3cm,arrow box,name=p,on chain=2,arrow box arrows={north:.5cm},minimum size=0.5cm] {}; 
 \draw[on chain=2]{}; 
 \node [draw, name=q,left=0cm,on chain=2]{q3}; 
 \draw [<-] (p) -- (q); 
 \chainin (q) [join]; 
 \end{scope} 
 \end{tikzpicture} 
 \end{figure} 	
 \end{frame} 
\begin{frame}[fragile]{CoMa Turingmaschine} 
 
 \begin{itemize} 
 \item Eingabealphabet : ['11'] 
 \item Bandalphabet : [1, 2, 3] 
 \item Leerzeichen : B 
 \item anzahl an Zustaenden : 2 
 \item akzept. Zustaende : q1,q2
 \end{itemize} 
 \begin{figure} 
 \begin{tikzpicture} 
 
 \edef\sizetape{0.7cm} 
 \tikzstyle{tmtape}=[draw,minimum size=\sizetape] 
 
 %% Draw TM tape 
 \begin{scope}[start chain=1 going right,node distance=-0.15mm] 
 \node [on chain=1,tmtape,draw=none] {$\ldots$}; 
 \node [on chain=1,tmtape] {B}; 
 \node [on chain=1,tmtape] {B}; 
 \node [on chain=1,tmtape] {B}; 
 \node [on chain=1,tmtape] {26}; 
 \node [on chain=1,tmtape] {78}; 
 \node [on chain=1,tmtape] {B}; 
 \node [on chain=1,tmtape] {B}; 
 \node [on chain=1,tmtape] {B}; 
 \node [on chain=1,tmtape] {B}; 
 \node [on chain=1,tmtape] {B}; 
 \node [on chain=1,tmtape] {B}; 
 \node [on chain=1,tmtape] {B}; 
 \node [on chain=1,tmtape] {B}; 
  
 \node [on chain=1,tmtape,draw=none] {$\ldots$}; 
 \end{scope} 
 \end{tikzpicture} 
 \begin{tikzpicture} 
 \node [draw,align=left]{akt. Zustand}; 
 \begin{scope}[start chain=2 going right] 
 \node [draw,left=3cm,arrow box,name=p,on chain=2,arrow box arrows={north:.5cm},minimum size=0.5cm] {}; 
 \draw[on chain=2]{}; 
 \node [draw, name=q,left=0cm,on chain=2]{q5}; 
 \draw [<-] (p) -- (q); 
 \chainin (q) [join]; 
 \end{scope} 
 \end{tikzpicture} 
 \end{figure} 	
 \end{frame} 
\begin{frame}[fragile]{CoMa Turingmaschine} 
 
 \begin{itemize} 
 \item Eingabealphabet : ['11'] 
 \item Bandalphabet : [1, 2, 3] 
 \item Leerzeichen : B 
 \item anzahl an Zustaenden : 2 
 \item akzept. Zustaende : q1,q2
 \end{itemize} 
 \begin{figure} 
 \begin{tikzpicture} 
 
 \edef\sizetape{0.7cm} 
 \tikzstyle{tmtape}=[draw,minimum size=\sizetape] 
 
 %% Draw TM tape 
 \begin{scope}[start chain=1 going right,node distance=-0.15mm] 
 \node [on chain=1,tmtape,draw=none] {$\ldots$}; 
 \node [on chain=1,tmtape] {B}; 
 \node [on chain=1,tmtape] {B}; 
 \node [on chain=1,tmtape] {B}; 
 \node [on chain=1,tmtape] {B}; 
 \node [on chain=1,tmtape] {1}; 
 \node [on chain=1,tmtape] {78}; 
 \node [on chain=1,tmtape] {B}; 
 \node [on chain=1,tmtape] {B}; 
 \node [on chain=1,tmtape] {B}; 
 \node [on chain=1,tmtape] {B}; 
 \node [on chain=1,tmtape] {B}; 
 \node [on chain=1,tmtape] {B}; 
 \node [on chain=1,tmtape] {B}; 
  
 \node [on chain=1,tmtape,draw=none] {$\ldots$}; 
 \end{scope} 
 \end{tikzpicture} 
 \begin{tikzpicture} 
 \node [draw,align=left]{akt. Zustand}; 
 \begin{scope}[start chain=2 going right] 
 \node [draw,left=3cm,arrow box,name=p,on chain=2,arrow box arrows={north:.5cm},minimum size=0.5cm] {}; 
 \draw[on chain=2]{}; 
 \node [draw, name=q,left=0cm,on chain=2]{q1}; 
 \draw [<-] (p) -- (q); 
 \chainin (q) [join]; 
 \end{scope} 
 \end{tikzpicture} 
 \end{figure} 	
 \end{frame} 
\begin{frame}[fragile]{CoMa Turingmaschine} 
 
 \begin{itemize} 
 \item Eingabealphabet : ['11'] 
 \item Bandalphabet : [1, 2, 3] 
 \item Leerzeichen : B 
 \item anzahl an Zustaenden : 2 
 \item akzept. Zustaende : q1,q2
 \end{itemize} 
 \begin{figure} 
 \begin{tikzpicture} 
 
 \edef\sizetape{0.7cm} 
 \tikzstyle{tmtape}=[draw,minimum size=\sizetape] 
 
 %% Draw TM tape 
 \begin{scope}[start chain=1 going right,node distance=-0.15mm] 
 \node [on chain=1,tmtape,draw=none] {$\ldots$}; 
 \node [on chain=1,tmtape] {B}; 
 \node [on chain=1,tmtape] {B}; 
 \node [on chain=1,tmtape] {B}; 
 \node [on chain=1,tmtape] {B}; 
 \node [on chain=1,tmtape] {B}; 
 \node [on chain=1,tmtape] {wtf}; 
 \node [on chain=1,tmtape] {1}; 
 \node [on chain=1,tmtape] {78}; 
 \node [on chain=1,tmtape] {B}; 
 \node [on chain=1,tmtape] {B}; 
 \node [on chain=1,tmtape] {B}; 
 \node [on chain=1,tmtape] {B}; 
 \node [on chain=1,tmtape] {B}; 
  
 \node [on chain=1,tmtape,draw=none] {$\ldots$}; 
 \end{scope} 
 \end{tikzpicture} 
 \begin{tikzpicture} 
 \node [draw,align=left]{akt. Zustand}; 
 \begin{scope}[start chain=2 going right] 
 \node [draw,left=3cm,arrow box,name=p,on chain=2,arrow box arrows={north:.5cm},minimum size=0.5cm] {}; 
 \draw[on chain=2]{}; 
 \node [draw, name=q,left=0cm,on chain=2]{q3}; 
 \draw [<-] (p) -- (q); 
 \chainin (q) [join]; 
 \end{scope} 
 \end{tikzpicture} 
 \end{figure} 	
 \end{frame} 
\end{document}