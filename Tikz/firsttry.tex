\documentclass[10pt]{beamer} \n
\hypersetup{pdfpagemode=FullScree
n} \n             \usepackage{tikz} \n             \usetikzlibrary{shadows,pat
terns,shapes} \n             \\usetikzlibrary{shapes.arrows,chains} \n
   % serifenfreier Font -- fuer Praesentation geeignet/er \n            \\renewc
ommand\\familydefault{\\sfdefault} \n             \\listfiles % damit im Log all
e benutzten Pakete aufgelistet werden \n             \\usetheme[progressbar=fram
etitle]{metropolis} \n             \\usepackage{appendixnumberbeamer} \n
     \\usepackage{booktabs} \n             \\usepackage[scale=2]{ccicons} \n
         \\usepackage[utf8]{inputenc} \n             \\usepackage{pgfplots} \n
           \\usepgfplotslibrary{dateplot} \n             \\usepackage[ngerman]{b
abel} \n             \\usepackage{xspace} \n             \\newcommand{\\themenam
e}{\\textbf{\\textsc{metropolis}}\\xspace} \n             \\title{Deterministisc
he Turing-Maschine} \n             \\subtitle{TU-Berlin, SoSe17, CoMa2 Programmi
erprojekt} \n             \\author{Carolin Schwarz 371802 CO2-155,\\\\ Duc Hoang
 Tran 222476 CO2-133,\\\\ Li Yinying 380390 CO2-144\\\\} \n             \\instit
ute{Betreuer: Ansgar} \n             \\titlegraphic{\\hfill\\includegraphics[hei
ght=2cm]{tu.jpg}} \n             \\begin{document} \n             \\maketitle \n
 \\begin{frame}[fragile]{CoMa Turingmaschine} \n \n                     \\begin{
itemize} \n                             \\item Eingabealphabet : \n
                \\item Bandalphabet : \n                             \\item Leer
zeichen :  \n                             \\item anzahl an Zustaenden : \n
                       \\item akzept. Zustaende : \n                     \\end{i
temize} \n                     \\begin{figure} \n                             \\
begin{tikzpicture} \n \n                             \\edef\\sizetape{0.7cm} \n
                            \\tikzstyle{tmtape}=[draw,minimum size=\\sizetape] \
n                             \n                             %% Draw TM tape \n
                            \\begin{scope}[start chain=1 going right,node distan
ce=-0.15mm] \n                                     \\node [on chain=1,tmtape,dra
w=none] {$\\ldots$}; \n                                     \node [on chain=1,tm
tape] {}; \n                                     "+self.get_bandtikz(listband)+"
 \n                                     \\node [on chain=1,tmtape,draw=none] {$\
\ldots$}; \n                             \\end{scope} \n
 \\end{tikzpicture} \n                         \\begin{tikzpicture} \n
                   \node [draw,align=left]{akt. Zustand}; \n
         \\begin{scope}[start chain=2 going right] \n
          \\node [draw,left=3cm,arrow box,name=p,on chain=2,arrow box arrows={no
rth:.5cm},minimum size=0.5cm] {}; \n                                     \\draw[
on chain=2]{}; \n                                     \\node [draw, name=q,left=
0cm,on chain=2]{q0}; \n                                     \\draw [<-] (p) -- (
q); \n                                     \\chainin (q) [join]; \n
                    \\end{scope} \n                         \\end{tikzpicture} \
n                     \\end{figure} \t\n                 \\end{frame} \n\\end{do
cument}'
